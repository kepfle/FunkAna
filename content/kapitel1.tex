\chapter{Beispiele normierter R\"aume}
Vektorr\"aume \"uber $ \K=\R $ oder $ \K=\C $. Den trivialen Vektorraum $ \lbrace 0\rbrace $ schlie\ss en wir aus.
\begin{definition}
Sei $ X $ ein $ \K- $Vektorraum. Eine Abbildung $ p\colon X\rightarrow[0,\infty[ $ hei\ss t \deftxt{Halbnorm}, falls
\begin{enumerate}
\item[a)] $ p(\lambda x)=|\lambda|p(x)\forall\lambda\in\K\forall x\in X $
\item[b)] $ p(x+y)\leq p(x)+p(y)\forall x,y\in X $ (Dreiecksungleichung)
\end{enumerate}
Gilt zus\"atzlich c) $ p(x)=0\Rightarrow x=0 $, so hei\ss t $ p $ eine \deftxt{Norm}. Das Paar $ (X,p) $ hei\ss t \deftxt{(halb-)normierter Raum}.\\
Ist $ p $ bekannt, so hei\ss t $ X $ \deftxt{(halb-)normierter Raum}. Normen werden mit $ \norm{\cdot} $ (statt $ p $) bezeichnet.
\end{definition}
\begin{bemerkung*}
\begin{enumerate}
\item[]
\item Aus a) folgt $ p(0)=0 $ (w\"ahle $ \lambda=0 $).
\item Sei $ (X,\norm{\cdot}) $ ein normierter Raum. Dann ist $ d(x,y)\coloneqq\norm{x-y}\forall x,y\in X $ eine \deftxt{Metrik auf $ X $}.
\end{enumerate}
\end{bemerkung*}
\newpage
\begin{definition}
Sei $ X $ ein normierter Raum.
\begin{enumerate}
\item[a)] $ (x_n)_{n\in\N})\subseteq X $ ist eine \deftxt{Cauchyfolge}, falls:
\[ \forall\e>0\exists N\in\N\forall n,m\geq N\colon\norm{x_n-x_m}<\e \]
\item[b)] $ (x_n)_{n\in\N} $ konvergiert gegen $ x\in X $, falls:
\[ \forall\e>0\exists N\in\N\forall n\geq N\colon\norm{x_n-x}<\e \]
\item[c)] $ X $ ist ein \deftxt{Banachraum}, wenn jede Cauchyfolge in $ X $ konvergiert.
\end{enumerate}
\end{definition}
\begin{bemerkung*}
In normierten R\"aumen ist jede konvergente Folge eine Cauchyfolge.
\end{bemerkung*}
\begin{beispiel*}
$ \K^n $ ist ein Banachraum mit jeder der folgenden Normen ($ x=(x_1,...,x_n) $):
\[ \norm{x}_1=\sum_{j=1}^{n} |x_j| \]
\[ \norm{x}_2=\sqrt{\sum_{j=1}^{n}|x_j|^2} \]
\[ \norm{x}_\infty=\max_{j=1,...,n}|x_j| \]
\end{beispiel*}
\begin{proposition}
In einem endlichdimensionalen Vektorraum $ X $ sind alle Normen \"aquivalent, d.h. zu je zwei Normen $ \norm{\cdot} $, $ \vertiii{\cdot}$ auf $ X $ gibt es eine Konstante $ c>0 $, so dass
\[ \frac{1}{c}\norm{x}\leq\vertiii{x}\leq c\norm{x}\forall x\in X \]
\end{proposition}
\newpage
\begin{beweis}
O.B.d.A. $ X=\K^n $.
\[ \norm{(x_1,...,x_n)}_1=\sum_{j=1}^{n}|x_j| \]
ist eine Norm auf $ X $. Sei $ \norm{\cdot} $ eine weitere Norm auf $ X $ und $ e_j\coloneqq (0,...,0,1_j,0,...,0) $, $ 1\leq j\leq n $.
\[ \norm{x}=\norm{\sum_{j=1}^{n}x_je_j}\leq\sum_{j=1}^{n}|x_j|\norm{e_j}\leq\underbrace{\left(\max_{j=1,...,n}\norm{e_j}\right)}_{\eqqcolon c}\sum_{j=1}^{n}|x_j|=c\norm{x}_1 \]
Also: $ \norm{x}\leq c\norm{x}_1 $ f\"ur ein $ c>0 $ und alle $ x\in X $ und $ \norm{\cdot}\colon (X,\norm{\cdot}_1)\rightarrow[0,\infty] $ ist stetig.\\
$ S=\lbrace x\in X\mid\norm{x}_1=1\rbrace $ kompakt ($ \Leftrightarrow $beschr\"ankt und abgeschlossen). Dann folgt: $ \min_{x\in S}\norm{x}=\delta>0 $ mit $ \delta=\norm{\tilde{x}} $ mit $ \norm{\tilde{x}}_1=1 $.
\[ \norm{x}_1=\frac{1}{\delta}\min_{\tilde{x}\in S}\norm{\tilde{x}}\norm{x}_1\leq\frac{1}{\delta}\norm{\frac{x}{\norm{x}_1}}\norm{x}_1=\frac{1}{\delta}\norm{x}_1 \]
\end{beweis}
%Vorlesung nachtragen
\begin{satz}[Minkowskische Ungleichung, Version f\"ur Folgen]
F\"ur $ x,y\in \ell^p $, $ 1\leq p<\infty $, gilt $ \norm{x+y}\leq\norm{x}_p+\norm{y}_p $.
\end{satz}
\begin{beweis}
$ p=1.\surd $ Sei also $ p>1$. Wir zeigen die \"aquivalente Ungleichung
\[ \norm{x+y}_p^p\leq(\norm{x}_p+\norm{y}_p)\norm{x+y}^{p-1}_p \]
Sei $ x=(x_n)_n $ und $ y=(y_n)_n $. Dann mit $ \frac{1}{p}+\frac{1}{q}=1 $ nach der H\"olderschen Ungleichung:
\begin{align*} \norm{x+y}_p^p&=\sum_{n\in\N}|x_n+y_n|^p\\&=\sum_{n\in\N}|x_n+y_n||x_n+y_n|^{p-1}\\&\leq\sum_{n\in\N}|x_n||x_n+y_n|^{p-1}+\sum_{n\in\N}|y_n||x_n+y_n|^{p-1}\\&\leq\left(\sum_{n\in\N}|x_n|^p\right)^\frac{1}{p}\left(\sum_{n\in\N}(|x_n+y_n|^{p-1})^q\right)^\frac{1}{q}+\left(\sum_{n\in\N}|y_n|^p\right)^\frac{1}{p}\left(\sum_{n\in\N}(|x_n+y_n|^{p-1})^q\right)^\frac{1}{q}\\&=\norm{x}_p\norm{x+y}_p^{\sfrac{p}{q}}+\norm{y}_p\norm{x+y}_p^{\sfrac{p}{q}}\\&=(\norm{x}_p+\norm{y}_p)\norm{x+y}_p^{p-1} \end{align*}
Da au\ss erdem gilt: $ \norm{\lambda x}_p=|\lambda|\norm{x}_p $ f\"ur $ \lambda\in\K $ und $ x\in\ell^p $ und $ \norm{x}_p=0\Leftrightarrow x=0 $, ist $ (\ell^p,\norm{\cdot}_p) $ ein normierter Raum.\\
Behauptung: $ (\ell^p,\norm{\cdot}_p) $ f\"ur $ 1\leq p<\infty $ ist vollst\"andig, d.h. ein Banachraum.
\begin{beweis}
Sei $ (x_n) $ eine Cauchyfolge in $ \ell^p $. Wir schreiben $ (x_n)=(x_m^{(n)})_{m\in\N} $, $ x_m^{(n)}\in\K $. F\"ur alle $ y=(y_m)_{m\in\N} $ und alle $ m\in\N $ gilt: $ |y_m|\leq\norm{y}_p $.
\[ (x_n)_n\text{ Cauchyfolge}\Leftrightarrow\forall\e>0\exists N\in\N\forall n,k\geq N:\norm{x_n-x_k}_p<\e \]
Aus
\[ |x_m^{(n)}-x_m^{(k)}|\leq\norm{x_n-x_k}_p \]
folgt: F\"ur beliebige $ m\in\N $ ist $ (x_m^{(n)})_{n\in\N} $ eine Cauchyfolge in $ \K $ ($ \K $ ist vollst\"andig).\\
Sei $ x_m\coloneqq\lim_{n\to\infty}x_m^{(n)} $ und $ x=(x_m)_{m\in\N} $. Es bleibt zu zeigen: $ x\in\ell^p $ und $ \norm{x_n-x}_p\xrightarrow{n\to\infty}0 $. Sei $ \e>0 $ beliebig. Dann existiert $ N\in\N:\norm{x_n-x_k}_p<\e\forall n,k\geq N $. Insbesondere gilt f\"ur $ M\in\N: $
\[ \left(\sum_{m=1}^M|x_m^{(n)}-x_m^{(k)}|^p\right)^\frac{1}{p}\leq\norm{x_n-x_k}_p<\e\forall n,k\geq N \]
Mit $ k\to\infty $ gilt $ \forall M\in\N\forall n\geq N $:
\[ \left(\sum_{m=1}^{M}|x_m^{(n)}-x_m|^p\right)^\frac{1}{p}\leq\e \]
$ M $ war beliebig und somit folgt:
\[ \left(\sum_{m=1}^{\infty}|x_m^{(n)}-x_m|^p\right)^\frac{1}{p}\leq\e\forall n\geq N \]
Somit ist $ x_N-x\in\ell^p $ und $ x=x-x_N+x_N\in\ell^p $ und $ \norm{x_n-x}_p\xrightarrow{n\to\infty}0 $.
\end{beweis} 
\end{beweis}
\begin{satz}
Sei $ (X,\norm{\cdot}^\ast) $ ein halbnormierter Raum.
\begin{enumerate}
\item[a)] $ N\coloneqq\lbrace x\in X\mid \norm{x}^\ast=0\rbrace $ ist ein Untervektorraum von $ X $.
\item[b)] $ \norm{[x]}\coloneqq\norm{x}^\ast $ definiert eine Norm auf $ X/N $.
\item[c)] Ist $ X $ vollst\"andig, d.h. in $ X $ konvergiert jede Cauchyfolge, so ist $ X/N $ ein Banachraum.
\end{enumerate}
\end{satz}
\begin{beweis}
\begin{enumerate}
\item[a)] $ \surd $
\item[b)] $ \norm{\cdot} $ ist wohldefiniert, d.h. unabh\"angig vom Repr\"asentanten der \"Aquivalenzklasse: Seien $ x,y\in X $ mit $ [x]=[y] $. Zu zeigen: $ \norm{x}^\ast=\norm{y}^\ast $.
\[ [x]=[y]\Leftrightarrow x\sim y\Leftrightarrow x-y\in N\Leftrightarrow\norm{x-y}^\ast=0 \]
Die umgekehrte Dreiecksungleichung \[ |\norm{x}^\ast-\norm{y}^\ast|\leq\norm{x-y}^\ast \] zeigt $ \norm{x}^\ast=\norm{y}^\ast $. Homogenit\"at und Dreiecksungleichung folgen nun direkt aus den entsprechenden Eigenschaften von $ \norm{\cdot}^\ast $.
\[ \norm{[x]}=0\Leftrightarrow\norm{x}^\ast=0\Leftrightarrow x\in N\Leftrightarrow x\sim 0\Leftrightarrow[x]=[0] \]
\item[c)] Folgt aus: 
\[ ([x_n])_{n\in\N}\text{ Cauchyfolge in }X/N\Leftrightarrow([x_n])_{n\in\N}\text{ Cauchyfolge in }X \]
\end{enumerate}
\end{beweis}
\begin{beispiel*}[Die $ L^p-$R\"aume]
Sei $ I $ ein Intervall, dann ist $ (I,B(I),\lambda_1) $ ein Ma\ss raum (gleiche \"Uberlegungen f\"ur einen Ma\ss raum $ (X,\sA,\mu) $). Sei $ \sL^\infty(I)\coloneqq\lbrace f\colon I\rightarrow\K\mid f\text{ messbar}, \exists N\in B(I)\text{ mit }\lambda_1(N)=0, f|_{I\setminus N} \text{ ist beschr\"ankt}\rbrace $.
\[ \norm{f}^\ast_{L^\infty}\coloneqq\inf_{\substack{N\in B(I)\\\lambda_1(N)=0}}\sup_{t\in I\setminus N}|f(t)|=\inf_{\substack{N\in B(I)\\\lambda_1(N)=0}}\norm{f|_{I\setminus N}}_\infty \]
Es gilt:
\begin{enumerate}
\item $ f\in\sL^\infty(I)\Rightarrow\norm{f}^\ast_{L^\infty}<\infty $
\item Zu $ f\in\sL^\infty(I) $ gibt es eine messbare Nullmenge $ N $ mit $ \norm{f}^\ast_{L^\infty}\coloneqq\norm{f|_{I\setminus N}}_\infty $.
\begin{beweis}
Zu $ r\in\N $ w\"ahlen wir eine messbare Nullmenge $ N_r $ mit $ \norm{f|_{I\setminus N_r}}_\infty\leq\norm{f}^\ast_{L^\infty}+\frac{1}{r} $. Dann ist $ N\coloneqq\bigcup_{r\in\N}N_r $ auch eine messbare Nullmenge und es gilt:
\[ \norm{f}_{L^\infty}^\ast\leq \norm{f|_{I\setminus N}}_\infty\leq\norm{f|_{I\setminus N_r}}_\infty\leq\norm{f}_{L^\infty}^\ast+\frac{1}{r}\forall r\in\N \]
Da $ r $ beliebig ist, folgt die Behauptung.
\end{beweis}
\item $ (\sL^\infty(I),\norm{\cdot}^\ast_{L^\infty}) $ ist ein halbnormierter Vektorraum.
\begin{beweis}
Wir zeigen nur die Dreiecksungleichung, alle weiteren Eigenschaften folgen direkt. Seien $ f_1, f_2\in\sL^\infty(I) $ und $ N_1, N_2 $ messbare Nullmengen gem\"a\ss\ ii).
\begin{align*} \norm{f_1+f_2}^\ast_{L^\infty}&=\inf_{\substack{N\in B(I)\\\lambda_1(N)=0}}\norm{(f_1+f_2)_{I\setminus N}}_\infty\\&\leq\norm{(f1+f_2)|_{I\setminus(N_1\cup N_2)}}_\infty\\&\leq\norm{f_1|_{I\setminus(N_1\cup N_2)}}_\infty+\norm{f_2|_{I\setminus(N_1\cup N_2)}}_\infty\\&\leq\norm{f_1|_{I\setminus N_1}}_\infty+\norm{f_2|_{I\setminus N_2}}_\infty\\&=\norm{f_1}^\ast_{L^\infty}+\norm{f_2}^\ast_{L^\infty} \end{align*}
\end{beweis}
\item $ (\sL^\infty(I),\norm{\cdot}^\ast_{L^\infty}) $ ist vollst\"andig, d.h. jede Cauchyfolge ist konvergent.
\begin{beweis}
Sei $ (f_n)_n $ eine Cauchyfolge in $ \sL^\infty(I) $. Nach ii) existieren messbare Nullmengen $ N_{n,m} $ mit
\[ \norm{f_n-f_m}_{L^\infty}^\ast=\norm{(f_n-f_m)|_{I\setminus N_{n,m}}}_\infty \]
Sei $ N=\bigcup_{n,m\in\N} N_{n,m} $ (abz\"ahlbare Vereinigung). Dies ist auch eine messbare Nullmenge und es gilt:
\[ \norm{f_n-f_m}^\ast_{L^\infty}=\norm{(f_n-f_m)|_{I\setminus N}}_\infty \]
Also ist $ (f_n|_{I\setminus N})_{n\in\N} $ eine Cauchyfolge im Banachraum $ (\ell^\infty(I\setminus N),\norm{\cdot}_\infty) $. Daher existiert $ g\in\ell^\infty(I\setminus N) $ und $ f_n|_{I\setminus N}\xrightarrow{n\to\infty}g $ in $ \ell^\infty(I\setminus N) $. Setze $ f(t)= \begin{cases}
g(t)&t\in I\setminus N\\0&t\in N
\end{cases} $. Dann ist $ f $ beschr\"ankt und als punktweiser Limes der messbaren Funktionenfolge $ (f_n\chi_{I\setminus N})_{n\in\N} $ wieder messbar. Daraus folgt:
\[ \norm{f_n-f}^\ast_{L^\infty}\leq\norm{(f_n-f)_{I\setminus N}}_\infty\xrightarrow{n\to\infty}0 \]
\end{beweis}
\end{enumerate}
\end{beispiel*}
%Vorlesung nachtragen
Sei $ N_p\coloneqq\lbrace f\in\sL^p(I)\mid\norm{f}_p^ast=0\rbrace =\lbrace f\colon I\rightarrow\K\mid f\text{ messbar und } f=0\text{ fast \"uberall}\rbrace $, $ L^p(I)=\sL^p(I)/N_p $, $ \norm{[f]}_p=\norm{f}_p^\ast $. Dann ist $ (L^p(I), \norm{\cdot}_p) $ ein Banachraum.
\begin{bemerkung*}
Ein nicht vollst\"andiger Raum $ X $ kann stets in einen Banachraum 'eingebettet' werden. Sei $ CF(X)\coloneqq\lbrace (x_n)_n\subset X\mid (x_n)_n\text{ ist eine Cauchyfolge}\rbrace $. Auf $ CF(X) $ definieren wir die \"Aquivalenzrelation
\[ (x_n)_n\sim(y_n)_n\Leftrightarrow\lim_{n\to\infty}\norm{x_n-y_n}=0 \]
Sei $ \hat X\coloneqq\lbrace [(x_n)]\mid (x_n)_n\subseteq CF(X)\rbrace $ mit $ \norm{[(x_n)_n]}=\lim_{n\to\infty}\norm{x_n} $. Dann gilt: $ (\hat X,\norm{\cdot}) $ ist ein Banachraum und indem man $ X $ mit den konstanten Folgen in $ \hat X $ identifiziert, wird $ X $ in nat\"urlicher Weise in $ \hat X $ dicht eingebettet (d.h. $ X\subset \hat X $ und $ \bar X=\hat X $).\\
$ \hat X $ nennt man auch die \deftxt{Vervollst\"andigung von $ X $}.
\end{bemerkung*}
\newpage
\begin{satz}
Sei $ X $ ein normierter Raum.
\begin{enumerate}
\item Aus $ x_n\rightarrow x $ in $ X $ und $ y_n\rightarrow y $ in $ X $ folgt $ x_n+y_n\rightarrow x+y $.
\item Aus $ \lambda_n\rightarrow\lambda $ in $ \K $ und $ x_n\rightarrow x $ in $ X $ folgt $ \lambda_nx_n\rightarrow \lambda x $.
\item Aus $ x_n\rightarrow x $ folgt $ \norm{x_n}\rightarrow \norm{x} $.
\end{enumerate}
\end{satz}
\begin{bemerkung*}
Aus iii) folgt: Konvergente Folgen in $ X $ sind beschr\"ankt.
\end{bemerkung*}
\begin{beweis}
\begin{enumerate}
\item folgt aus \[ \norm{x_n+y_n-(x+y)}\leq\norm{x_n-x}+\norm{y_n-y}\rightarrow 0 \]
\item 
\[ \norm{\lambda_nx_n-\lambda x}=\norm{\lambda_n(x_n-x)+(\lambda_n-\lambda)x}\leq|\lambda_n|\norm{x_n-x}+|\lambda_n-\lambda|\norm{x} \]
\item folgt aus
\[ 0\leq|\norm{x_n}-\norm{x}|\leq\norm{x_n-x}\rightarrow 0 \]
\end{enumerate}
\end{beweis}
\begin{satz}
Ist $ U $ ein Untervektorraum des normierten Raumes $ X $, so ist sein Abschluss $ \bar U $ ebenfalls ein Untervektorraum.
\end{satz}
\begin{beweis}
Seien $ x,y\in\bar U $. Dann existieren Folgen $ (x_n)_n $, $ (y_n) $ in $ U $ mit $ x_n\rightarrow x $ und $ y_n\rightarrow y $. Also:
\[ x_n+y_n\rightarrow x+y\Rightarrow x+y\in\bar U \]
Sei $ \lambda\in\K $ und $ x\in \bar U$. Dann existiert eine Folge $ (x_n)_n $ in $ U $ mit $ x_n\rightarrow x $. Es folgt:
\[ \lambda x_n\rightarrow \lambda x\Rightarrow \lambda x\in\bar U \]
\end{beweis}
\newpage
\begin{bemerkung*}
Ist $ \dim U<\infty $, dann ist $ U $ abgeschlossen. Im Allgemeinen ist ein Untervektorraum nicht abgeschlossen.
\end{bemerkung*}
\begin{satz}
Seien $ \norm{\cdot} $ und $ \vertiii{\cdot} $ zwei Normen auf $ X $. Dann sind \"aquivalent:
\begin{enumerate}
\item $ \norm{\cdot} $ und $ \vertiii{\cdot} $ sind \"aquivalent, d.h. $ \exists c_1,c_2>0:c_1\norm{x}\leq\vertiii{x}\leq c_2\norm{x}\forall x\in X $.
\item Eine Folge ist bez\"uglich $ \norm{\cdot} $ konvergent genau dann, wenn sie bez\"uglich $ \vertiii{\cdot} $ konvergent ist.
\item Eine Folge ist eine $ \norm{\cdot}- $Nullfolge genau dann, wenn sie eine $ \vertiii{\cdot}- $Nullfolge ist.
\end{enumerate}
\end{satz}
\begin{beweis}
i)$ \Rightarrow $ii)$ \Rightarrow $iii) klar. Es bleibt zu zeigen: iii)$ \Rightarrow $i).\\
Angenommen es gibt kein $ c_2>0 $, so dass die Ungleichung $ \vertiii{x}\leq c_2\norm{x}\forall x\in X $. Dann gilt f\"ur alle $ n\in\N $: $ \exists x_n\in X:\vertiii{x_n}>n\norm{x_n} $. Setze $ y_n\coloneqq\frac{1}{n}\frac{x_n}{\norm{x_n}} $. Dann folgt:
\[ \norm{y_n}=\norm{\frac{1}{n}\frac{x_n}{\norm{x_n}}}=\frac{1}{n}\rightarrow 0 \]
Also ist $ (y_n)_n $ eine $ \norm{\cdot}- $Nullfolge und mit iii) somit auch eine $ \vertiii{\cdot}- $Nullfolge. Aber \[ \vertiii{y_n}=\vertiii{\frac{1}{n}\frac{x_n}{\norm{x_n}}}=\frac{1}{n\norm{x_n}}\vertiii{x_n}>\frac{n\norm{x_n}}{n\norm{x_n}}=1\lightning \]
Die Existenz von $ c_1>0:c_1\norm{x}\leq\vertiii{x} \forall x\in X$ l\"asst sich analog zeigen. 
\end{beweis}
\begin{bemerkung*}
Zus\"atzliche \"Aquivalenz:
\begin{enumerate}
\item[iv)] $ (X,\norm{\cdot}) $ und $ (X,\vertiii{\cdot}) $ besitzen die selben Cauchyfolgen. Somit: $ (X,\norm{\cdot}) $ ist vollst\"andig$ \Leftrightarrow (X,\vertiii{\cdot})$ ist vollst\"andig.
\end{enumerate}
\begin{beispiel*}
Aufgabe 3 zeigt, dass $ \norm{\cdot}_\infty $ und $ \norm{\cdot} $ auf $ C^1[a,b] $ nicht \"aquivalent sind.
\end{beispiel*}
\end{bemerkung*}
\begin{lemma}[Rieszsches Lemma]
Sei $ U $ ein abgeschlossener Unterraum des normierten Raums $ X $ mit $ U\neq X $. Ferner sei $ 0<\delta<1 $. Dann existiert ein $ x_\delta\in X $ mit $ \norm{x_\delta}=1 $ und
\[ \forall u\in U:\norm{x_\delta-u}\geq 1-\delta \]
\end{lemma}
\begin{beweis}
Sei $ x\in X\setminus U $.
\[ d\coloneqq\inf\lbrace\norm{x-u}\mid u\in U\rbrace>0 \]
Denn andernfalls g\"abe es eine Folge $ (u_n)_n\in U $ mit $ u_n\rightarrow x $ und $ x $ l\"age dann in $ \bar U=U $ (da $ U $ abgeschlossen). Es gilt: $ d<\frac{d}{1-\delta} $. Dann existiert ein $ u_\delta\in U $, f\"ur das gilt:
$ \norm{x-u_\delta}<\frac{d}{1-\delta} $.\\
Setze $ x_\delta\coloneqq\frac{x-u_\delta}{\norm{x-u_\delta}} $. Dann ist $ \norm{x_\delta}=1 $ und es gilt f\"ur $ u\in U $ beliebig:
\[ \norm{x_\delta-u}=\norm{\frac{x-u_\delta}{\norm{x-u_\delta}}-u}=\frac{1}{\norm{x-u_\delta}}\norm{x-(u_\delta+\norm{x-u_\delta}u)}\geq\frac{1}{\norm{x-u_\delta}}d>1-\delta \]
\end{beweis}
\begin{bemerkung*}
Das Rieszsche Lemma gilt nicht f\"ur $ \delta=0 $.
\begin{beispiel*}
Sei $ X=\lbrace x\in C[0,1]\mid x(1)=0\rbrace $. $ (X,\norm{\cdot}_\infty) $ ist ein normierter Raum. $ U=\left\lbrace x\in X\middle| \int_0^1x(t)\dd t=0\right\rbrace $ ist ein abgeschlossener Untervektorraum.\\
Angenommen es gibt ein Element $ x\in X $ mit $ \norm{x-u}_\infty\geq 1=\norm{x}_\infty\forall u\in U $. Setze $ x_n(t)=1-t^n $. Dann sind $ x_n\in X $, $ \norm{x_n}_\infty=1 $ und $ \int_0^1x_n(t)\dd t=1-\frac{1}{n+1} $. Setze
\[ \lambda_n=\frac{\int_0^1 x(t)\dd t}{1-\frac{1}{n+1}},\qquad u_n=x-\lambda_nx_n\in U \]
Daraus folgt: $ \norm{x-u_n}_\infty\geq 1 $ und $ \norm{x-u_n}_\infty=\norm{\lambda_nx_n}_\infty=|\lambda_n|\geq 1 $.
\[ \left|\int_0^1 x(t)\dd t\right|=|\lambda_n|\left|1-\frac{1}{n+1}\right|\geq\left|1-\frac{1}{n+1}\right|\geq 1 n\forall\in\N \]
Aber $ x\colon[0,1]\rightarrow\K $ stetig, $ \norm{x}_\infty\leq 1 $ und $ x(1)=0 $.
\end{beispiel*}
\end{bemerkung*}
%Vorlesung nachtragen
\begin{beweis}
$ [a,b]=[0,1] $. Sei $ f\in C[0,1] $.
\[ P_n(s)=B_n(s,f)\coloneqq\sum_{i=0}^{n}\binom{n}{i}s^i(1-s)^{n-i}f\left(\frac{i}{n}\right) \]
Zu zeigen: $ \norm{P_n-f}_\infty\rightarrow 0 $. Da $ f $ gleichm\"assig stetig ist, existiert zu $ \e>0 $ ein $ \delta>0 $ mit $ |s-t|\leq\sqrt{\delta} $ und es folgt $ |f(s)-f(t)|\leq\e $.\\
Es gilt f\"ur $ |s-t|>\delta $ mit $ \alpha=\frac{2\norm{f}_\infty}{\delta} $:
\[ |f(s)-f(t)|\leq|f(s)|+|f(t)|\leq 2\norm{f}_\infty=\alpha\delta\leq\alpha(s-t)^2 \]
Somit gilt f\"ur beliebige $ s,t\in[0,1] $:
\[ |f(s)-f(t)|\leq\alpha(s-t)^2+\e \]
Setze $ y_t(s)\coloneqq(t-s)^2 $. Dann folgt:
\[ -\e-\alpha y_t(s)<f(s)-f(t)<\alpha y_t(s)+\e\forall s,t\in[0,1] \]
Wir bestimmen nun die Bernstein-Polynome zu $ f_j(s)=s^j $ f\"ur $ j=0,1,2 $:
\[ B_n(s,f_0)=\sum_{i=0}^{n}\binom{n}{i}s^i(1-s)^{n-i}=(s+(1-s))^n=1 \]
\[ B_n(s,f_1)=\sum_{i=1}^{n}\binom{n}{i}s^i(1-s)^{n-i}\frac{i}{n}=\footnote{$ \binom{n}{i}\frac{i}{n}=\binom{n-1}{i-1} $}\sum_{i=0}^{n-1}\binom{n-1}{i}s^{i+1}(1-s)^{n-(i+1)}=s(s-(1-s))^{n-1}=s \]
\begin{align*} B_n(s,f_2)&=\sum_{i=0}^{n}\binom{n}{i}s^i(1-s)^{n-1}\left(\frac{i}{n}\right)^2\\&=\sum_{i=1}^{n}\binom{n-1}{i-1}s^i(1-s)^{n-i}\frac{i}{n}\\&=\sum_{i=0}^{n-1}\binom{n-1}{i}s^{i+1}(1-s)^{n-(i+1)}\frac{i+1}{n}\\&=\frac{s}{n}+\sum_{i=0}^{n-1}\binom{n-1}{i}s^{i+1}(1-s)^{n-(i+1)}\frac{i}{n}\\&=\frac{s}{n}+s\frac{n-1}{n}\underbrace{\sum_{i=0}^{n-1}\binom{n-1}{s}s^i(1-s)^{(n-1)-i}\frac{i}{n-1}}_{B_{n-1}(s,f_1)}\\&=\frac{s}{n}+s^2\frac{n-1}{n}\\&=s^2+\frac{s}{n}-\frac{s^2}{n}\\&=s^2+\frac{s(1-s)}{n} \end{align*}
Es folgt:
\[ -B_n(s,\e+\alpha y_t)=B_n(s,-\e-\alpha y_t)\leq B_n(s,f-f(t))\leq B_n(s,\alpha y_t+\e) \]
F\"ur alle $ s,t\in[0,1] $ gilt dann:
\begin{align*} |P_n(s)-f(t)|&=|B_n(s,f)-f(t)B_n(s,f_0)|\\&=|B_n(s,f)-B_n(s,f(t))|\\&=|B_n(s,f-f(t))|\leq B_n(s,\alpha y-t+\e)\\&=B_n(s,\e+\alpha(t^2-2st+t^2))=B_n(s,\e)+\alpha B_n(s,t^2)-2\alpha B_n(s,st)+\alpha B_n(s,s^2)\\&=\e+\alpha t^2-2\alpha ts+\alpha\left(s^2+\frac{s(1-s)}{n}\right)  \end{align*}
Mit $ s=t $ folgt dann:
\[ |P_n(t)-f(t)|\leq\e+\alpha t^2-2\alpha t^2+\alpha\left(t^2+\frac{t(i-1)}{n}\right)\leq\e\frac{\alpha}{n} \]
Also:
\[ \norm{P_n-f}_\infty\leq\e+\frac{\alpha}{n} \]
Hieraus folgt die gleichm\"assige Konvergenz.
\end{beweis}
\begin{korollar}
$ C[a,b] $ ist separabel.
\end{korollar}
\begin{beweis}
	Aus dem Approximationssatz folgt: $ C[a,b]=\overline{\text{lin}\lbrace x_n\mid n\in\N_0\rbrace} $ mit $ x_n(t)=t^n $.
\end{beweis}
\begin{satz}
	Sei $ 1\leq p<\infty $. $ C[a,b] $ ist dicht in $ L^p[a,b] $.
\end{satz}
\begin{beweis}
	Zu zeigen: $ \overline{C[a,b]}=L^p[a,b] $ (Abschluss bez\"uglich $ \norm{\cdot}_p $)\\
	Sei $ B([a,b]) $ die $ \sigma- $Algebra der Borelmengen auf $ [a,b] $. Aus der Definition des Lebesgueintegrals folgt: $ \text{lin}\lbrace\chi_A\mid A\in B([a,b])\rbrace $, der Raum der Stufenfunktionen, liegt dicht in $ L^p[a,b] $. Das Lebesgzema\ss\ ist \deftxt{regul\"ar}, d.h.:
	\[ \lambda(A)=\inf\lbrace\lambda(O)\mid A\subseteq O, O\text{ offen}\rbrace  \]
	Daraus folgt f\"ur alle $ A\in B([a,b]) $, dass f\"ur alle $ \e>0 $ eine offene Menge $ O $ mit $ A\subseteq O $ existiert so dass:
	\[ \norm{\chi_A-\chi_O}_p=\norm{\chi_{O\setminus A}}_p=\lambda(O\setminus A)^\frac{1}{p}<\e \]
	Somit:
	\[ \lbrace\chi_A\mid A\in B([a,b])\rbrace\subseteq\overline{\lbrace\chi_O\mid O\text{ offen}\rbrace} \]
	\[ L^p[a,b]=\overline{\text{lin}\lbrace\chi_A\mid A\in B([a,b])\rbrace}=\overline{\text{lin}\lbrace\chi_O\mid O\text{ offen}\rbrace} \]
	Jede offene Menge $ O $ ist eine abz\"ahlbare Vereinigung von paarweise disjunkten Intervallen $ I_j $. Aus $ \lambda(O)=\sum_{j=1}^{\infty}\lambda(I_j) $ folgt:
	$\chi_O\in\overline{\text{lin}\lbrace\chi_I\mid I\text{ offenes Intervall}\rbrace}$. Hieraus folgt nun: \[ L^p[a,b]=\overline{\text{lin}\lbrace\chi_I\mid I\text{ offenes Intervall}\rbrace} \]
	Es gen\"ugt zu zeigen: Zu jedem offenen Intervall $ I\subset[a,b] $ und jedem $ \e>0 $ existiert eine stetige Funktioen $ f $ mit $ \norm{f-\chi_I}_p<\e $. Sei $ \e>0 $ und $ a\leq a'<b'\leq b $. W\"ahle $ f(x) $ geeignet. Dann folgt, dass $ C[a,b] $ dicht in $ L^p[a,b] $ liegt.
\end{beweis}
\begin{korollar}
	$ 1\leq p<\infty $. $ L^p $ ist separabel.
\end{korollar}
\begin{beweis}
	Es gen\"ugt zu zeigen, dass die Polynome dicht in $ L^p[a,b] $ liegen. Sei $ f\in L^p[a,b] $. Nach \ref{satz1.21} existiert eine Folge $ (f_n)_n $ stetiger Funktionen mit $ \norm{f_n-f}_p\rightarrow 0 $. Nach dem Weierstra\ss schen Approximationssatz existieren Polynome $ P_n $ mit $ \norm{f_n-P_n}_\infty\leq\frac{1}{n} $. Wegen
	\[ \norm{g}_p=\left(\int_a^b |g|^p\dd\lambda\right)^\frac{1}{p}\leq(b-a)^\frac{1}{p}\norm{g}_\infty \]
	f\"ur $ g\in C[a,b] $ folgt $ \norm{f_n-P_n}_p\rightarrow 0 $ f\"ur $ n\rightarrow\infty $. Also folgt:
	\[ \norm{P_n-f}_p\leq\norm{P_n-f_n}_p+\norm{f_n-f}_p\rightarrow 0 \]
\end{beweis}
\begin{bemerkung*}
	Ohne Beweis sei noch erw\"ahnt:
	\begin{enumerate}
		\item $ T $ kompakter Raum$ \Rightarrow(C(T),\norm{\cdot}_\infty) $ ist separabel.
		\item $ \Omega $ offene Menge (z.B. $ \R $). $ L^p(\Omega) $ ist separabel, $ 1\leq p<\infty $.
	\end{enumerate}
\end{bemerkung*}
\begin{definition}
	Sei $ X $ ein normierter Raum und $ A\subseteq X $. Der  \deftxt{Abstand} von $ x\in X $ zu $ A $ ist gegeben durch:
	\[ d(x,A)\coloneqq\inf\lbrace\norm{x-a}\mid a\in A\rbrace \]
\end{definition}
\begin{bemerkung*}
 Es gilt: \[ d(x,A)=0\Leftrightarrow x\in\bar A \]
\end{bemerkung*}
\begin{satz}
	Sei $ X $ ein normierter Raum und $ U\subseteq X $ ein Untervektorraum. $ X/U $ bezeichnet die Menge der \"Aquivalenzklassen bez\"uglich der \"Aquivalenzrelation $ x\sim y\Leftrightarrow x-y\in U $. F\"ur $ x\in X $ sei $ [x]=x+U\in X/U $ die zugeh\"orige \"Aquivalenzklasse. Es gilt:
	\begin{enumerate}
		\item $ \norm{x}=d(x,U) $ definiert eine Halbnorm auf $ X/U $.
		\item Ist $ U $ abgeschlossen, so ist $ \norm{\cdot} $ eine Norm auf $ X/U $.
		\item Ist $ X $ vollst\"andig und $ U $ abgeschlossen, so ist $ X/U $ ein Banachraum.
	\end{enumerate}
\end{satz}
\begin{beweis}
	\begin{enumerate}
		\item $ \norm{\cdot} $ ist wohldefiniert, denn: $ [x_1]=[x_2] $ impliziert $ x_1=x_2+u $ f\"ur ein $ u\in U $, also $ d(x_1,U)=d(x_2,U) $.
		\begin{align*} \norm{\lambda[x]}&=\norm{[\lambda x]}\\&=d(\lambda x, U)\\&=\inf\lbrace \norm{\lambda x-u}\mid u\in U\rbrace\\&=\inf\lbrace\norm{\lambda x-\lambda u}\mid u\in U\rbrace\\&=\inf\lbrace|\lambda|\norm{x-u}\mid u\in U\rbrace\\&=|\lambda|d(x,U)\\&=|\lambda|\norm{x} \end{align*}
		Seien $ x_1,x_2\in X $, sei $ \e>0 $. Es existieren $ u_1,u_2\in U $ mit \[ \norm{x_i-u_i}\leq\norm{[x_i]}+\e, i=1,2 \]
		\[ \norm{[x_1]+[x_2]}=\inf\lbrace\norm{x_1+x_2-u}\mid u\in U\rbrace\leq\norm{x_1+x_2-(u_1+u_2)}\leq\norm{x_i-u_i}+\norm{x_2-u_2}\leq\norm{[x_1]}+\norm{[x_2]}+2\e \]
		Da $ \e>0 $ beliebig, gilt:
		\[ \norm{[x_1]+[x_2]}\leq\norm{[x_1]}+\norm{[x_2]} \]
		$ \norm{[0]}=d(0,U)=0 $, da $ 0\in U $.
		\item \[ \norm{[0]}=0\Leftrightarrow d(x,U)=0\Leftrightarrow x\in\bar U=U\Leftrightarrow[x]=[0] \]
		\item Wir benutzen \ref{lemma1.11}. Sei also $ (x_k)_k $ eine Folge in $ X $ mit $ \sum_{k=1}^{\infty}\norm{[x_k]}<\infty $. Zu zeigen: $ \sum_{k=1}^{\infty}[x_k] $ konvergiert in $ X/U $.\\
		O.B.d.A.: $ \norm{x_k}\leq\norm{[x_k]}+2^{-k} $.
		\[ \sum_{k=1}^{\infty}\norm{x_k}\leq\sum_{k=1}^{\infty}\norm{[x_k]}+\sum_{k=1}^{\infty}2^{-k}<\infty \]
		Nun folgt, da $ X $ vollst\"andig ist, mit \ref[lemma1.11]:
		\[ \exists x=\sum_{k=1}^{\infty}x_k\in X \]
		\[ \norm{[x]-\sum_{k=1}^{n}[x_k]}=\norm{\left[x-\sum_{k=1}^{n}x_k\right]}\leq\norm{x-\sum_{k=1}^{n}x_k}\xrightarrow{n\to\infty}0 \]
		Also: $ [x]=\sum_{k=1}^{\infty}[x_k] $
	\end{enumerate}
\end{beweis}
\begin{beispiel*}
	Sei $ D\subseteq[0,1] $ abgeschlossen. Wir betrachten den Quotienten $ C[0,1]/U $ mit $ U\coloneqq\lbrace x\in C[0,1]\mid x|_D=0\rbrace $. Die Quotientenabbildung ($ x\in C[0,1]\mapsto[x]\in C[0,1]/U $) identifiziert Funktionenm die auf $ D $ \"ubereinstimmen. Die Elemente von $ C[0,1]/U $ k\"onnen als Funktionen auf $ D $ angesehen werden.
\end{beispiel*}